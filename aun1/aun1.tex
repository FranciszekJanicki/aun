\documentclass[11pt]{article}
\usepackage[a4paper, margin=2.5cm]{geometry}
\usepackage{graphicx}
\usepackage{caption}
\usepackage{pdfcomment}
\usepackage{float}
\usepackage{tikz}
\captionsetup{font=small, skip=6pt}
\usepackage{titlesec}
\usepackage{parskip}
\setlength{\parskip}{4pt}
\setlength{\textfloatsep}{10pt}
\setlength{\floatsep}{6pt}
\setlength{\intextsep}{10pt}

\titlespacing*{\section}{0pt}{2ex plus .2ex minus .2ex}{1ex plus .2ex}
\titlespacing*{\subsection}{0pt}{1ex plus .1ex minus .1ex}{0.5ex plus .1ex}

    
\usepackage{parskip}
\setlength{\parskip}{2pt}

\title{Wzmacniacze mocy w napędzie elektrycznym. Przekształtnik tranzystorowy i tyrystorowy.}

\author{
  Mateusz Jaworski \\
  Piotr Migdalek \\
  Pawel Michalski \\
  Jakub Jaszczak \\
  Franciszek Janicki
}

\begin{document}

\maketitle

\section{Część I}

\subsection{WZMACNIACZ LINIOWY / WZMACNIACZ IMPULSOWY}

a) Badanie sprawnosci wzmacniacza liniowego BJT

\begin{figure}[H]
\centering
\includegraphics[width=0.8\textwidth]{aun1_liniowy_bjt.png}
\caption{Schemat pomiarowy wzmacniacza liniowego BJT}
\end{figure}

\begin{table}[H]
\centering
\begin{tabular}{|c|c|c|c|c|}
\hline
\textbf{Vcc [V]} & \textbf{V\_in [V]} & \textbf{P\_in + P\_vcc [mW]} & \textbf{P\_out [W]} & \textbf{Sprawność [\%]} \\
\hline
24 & 2  & 0{,}667  & 0{,}014683 & 2{,}201 \\
\hline
24 & 4  & 1{,}525  & 0{,}070000 & 4{,}590 \\
\hline
24 & 6  & 2{,}380  & 0{,}166000 & 6{,}975 \\
\hline
24 & 8  & 3{,}242  & 0{,}360000 & 11{,}104 \\
\hline
24 & 10 & 4{,}097  & 0{,}477000 & 11{,}643 \\
\hline
24 & 12 & 4{,}950  & 0{,}690000 & 13{,}939 \\
\hline
24 & 14 & 5{,}800  & 0{,}941000 & 16{,}224 \\
\hline
24 & 16 & 6{,}640  & 1{,}220000 & 18{,}373 \\
\hline
24 & 18 & 7{,}490  & 1{,}540000 & 20{,}561 \\
\hline
24 & 20 & 8{,}330  & 1{,}940000 & 23{,}289 \\
\hline
\end{tabular}
\caption{Sprawnosc pracy wzmacniacza liniowego BJT}
\end{table}

b) Badanie sprawnosci wzmacniacza impulsowego BJT, porownanie z wzmacniaczem liniowym BJT

\begin{figure}[H]
\centering
\includegraphics[width=0.8\textwidth]{aun1_impulsowy_bjt.png}
\caption{Schemat pomiarowy wzmacniacza impulsowego BJT}
\end{figure}

\begin{table}[H]
\centering
\begin{tabular}{|c|c|c|c|c|}
\hline
\textbf{Vcc [V]} & \textbf{V\_in [V]} & \textbf{P\_vcc + P\_in [W]} & \textbf{P\_out [W]} & \textbf{Sprawność [\%]} \\
\hline
24 & 2  & 0{,}54825 & 0{,}499   & 91{,}017 \\
\hline
24 & 4  & 1{,}054   & 0{,}974   & 92{,}410 \\
\hline
24 & 6  & 1{,}549   & 1{,}453   & 93{,}802 \\
\hline
24 & 8  & 2{,}046   & 1{,}928   & 94{,}233 \\
\hline
24 & 10 & 2{,}544   & 2{,}4023  & 94{,}430 \\
\hline
24 & 12 & 3{,}0476  & 2{,}882   & 94{,}566 \\
\hline
24 & 14 & 3{,}545   & 3{,}356   & 94{,}669 \\
\hline
24 & 16 & 4{,}042   & 3{,}831   & 94{,}780 \\
\hline
24 & 18 & 4{,}546   & 4{,}311   & 94{,}831 \\
\hline
24 & 20 & 5{,}043   & 4{,}833   & 95{,}836 \\
\hline
\end{tabular}
\caption{Sprawnosc pracy wzmacniacza impulsowego BJT}
\end{table}

\begin{figure}[H]
\centering
\includegraphics[width=0.8\textwidth]{aun1_liniowy_impulsowy_bjt.pdf}
\caption{Wykres sprawnosci wzmacniacza liniowego i impulsowego BJT}
\end{figure}

\subsection{TECHNOLOGIA BJT/MOSFET}

c) Badanie sprawności wzmacniacza impulsowego MOSFET i porównanie z wzmacniaczem impulsowym BJT oraz porównanie wpływu Rds_on i ładunku bramki na sprawność

\begin{figure}[H]
\centering
\includegraphics[width=0.8\textwidth]{aun1_impulsowy_mosfet.png}
\caption{Schemat pomiarowy wzmacniacza impulsowego MOSFET}
\end{figure}

\begin{table}[H]
\centering
\begin{tabular}{|c|c|c|c|c|}
\hline
\textbf{Vcc [V]} & \textbf{V\_in [V]} & \textbf{P\_vcc + P\_in [mW]} & \textbf{P\_out [W]} & \textbf{Sprawność [\%]} \\
\hline
24 & 2  & 0{,}499    & 0{,}4956   & 99{,}319 \\
\hline
24 & 4  & 0{,}95412  & 0{,}9506   & 99{,}631 \\
\hline
24 & 6  & 1{,}461    & 1{,}4575   & 99{,}760 \\
\hline
24 & 8  & 1{,}939    & 1{,}9355   & 99{,}819 \\
\hline
24 & 10 & 2{,}417    & 2{,}4135   & 99{,}855 \\
\hline
24 & 12 & 2{,}9009   & 2{,}8974   & 99{,}879 \\
\hline
24 & 14 & 3{,}379    & 3{,}375    & 99{,}882 \\
\hline
24 & 16 & 3{,}857    & 3{,}853    & 99{,}896 \\
\hline
24 & 18 & 4{,}3409   & 4{,}3373   & 99{,}917 \\
\hline
24 & 20 & 5{,}760    & 5{,}759    & 99{,}983 \\
\hline
\end{tabular}
\caption{Sprawnosc pracy wzmacniacza impulsowego MOSFET}
\end{table}

\begin{figure}[H]
\centering
\includegraphics[width=0.8\textwidth]{aun1_imp_bjt_vs_mosfet.pdf}
\caption{Wykres sprawnosci wzmacniacza impulsowego BJT i MOSFET}
\end{figure}

\begin{table}[H]
\centering
\begin{tabular}{|c|c|c|}
\hline
\textbf{Parametr} & \textbf{IRFL4310} & \textbf{IRF2907Z} \\
\hline
Rds\_on & 200\,m\(\Omega\) & 3,5\,m\(\Omega\) \\
\hline
GateCharge & 28\,nC & 180\,nC \\
\hline
\end{tabular}

\caption{Porównanie parametrów tranzystorów IRFL4310 i IRF2907Z}
\end{table}

\begin{table}[H]
\centering
\begin{tabular}{|c|c|c|c|c|c|c|}
\hline
\textbf{f [Hz]} & \multicolumn{3}{c|}{\textbf{IRFL4310}} & \multicolumn{3}{c|}{\textbf{IRF2907Z}} \\
\cline{2-7}
 & $P_{in}+P_{vcc}$ [W] & $P_{out}$ [W] & Sprawność [\%] & $P_{in}+P_{vcc}$ [W] & $P_{out}$ [W] & Sprawność [\%] \\
\hline
1k   & 2,9009 & 2,8836 & 99,41 & 2,9009 & 2,8988 & 99,93 \\
\hline
2k   & 2,8917 & 2,8916 & 99,99 & 2,9219 & 2,9177 & 99,85 \\
\hline
4k   & 2,9078 & 2,9076 & 99,99 & 2,9638 & 2,9555 & 99,72 \\
\hline
8k   & 2,9400 & 2,9397 & 99,96 & 3,0478 & 3,0312 & 99,45 \\
\hline
16k  & 3,1335 & 3,1321 & 99,96 & --- & -- & --- \\
\hline
32k  & 3,1332 & 3,1319 & 99,92 & --- & --- & --- \\
\hline
64k  & 3,3909 & 3,3882 & 99,92 & --- & --- & --- \\
\hline
\end{tabular}
\caption{Porównanie sprawności tranzystorów IRFL4310 oraz IRF2907Z przy 50\% duty cycle. Dla częstotliwości PWM powyżej 64 kHz tranzystor IRFL4310 zaczyna nie nadążać z domykaniem oraz otwieraniem, natomiast dla częstotliwości 16 kHz robi to IRF2907Z.}
\end{table}

\begin{figure}[H]
\centering
\includegraphics[width=0.8\textwidth]{aun1_imp_mosfet_comparison.pdf}
\caption{Wykres sprawnosci wzmacniaczy impulsowych MOSFET}
\end{figure}

\subsection{ZJAWISKA W OBWODZIE D-S TRANZYSTORA WYNIKAJĄCE Z PARAMETRÓW PASOŻYTNICZYCH OBWODU}

d) Badanie wartosci pasozytniczych w ukladzie RLD sterowanym tranzystorem MOSFET

\begin{figure}[H]
\centering
\includegraphics[width=0.8\textwidth]{aun1_rld_without_snubber.png}
\caption{Schemat pomiarowy ukladu RLD sterowanego tranzystorem MOSFET}
\end{figure}

\begin{figure}[H]
\centering
\includegraphics[width=0.8\textwidth]{aun1_rld_without_snubber_rgate100ohm.png}
\caption{Przebieg napiecia na wyjsciu tranzystora MOSFET sterujacego ukladem RLD z rezystancja bramki 100 [Ohm]}
\end{figure}

\begin{figure}[H]
\centering
\includegraphics[width=0.8\textwidth]{aun1_rld_without_snubber_rgate10ohm.png}
\caption{Przebieg napiecia na wyjsciu tranzystora MOSFET sterujacego ukladem RLD z rezystancja bramki 10 [Ohm]}
\end{figure}

e) Dobor tlumika oscylacji napiecia (parametry snubber'a)

\begin{figure}[H]
\centering
\includegraphics[width=0.8\textwidth]{aun1_rld_with_snubber.png}
\caption{Schemat pomiarowy ukladu RLD sterowanego tranzystorem MOSFET z tlumikiem oscylacji w obwodzie D-S tranzystora}
\end{figure}

Dobór parametrów tłumika (snubbera):
\begin{enumerate}
  \item Mierzymy częstotliwość oscylacji:
  \[
  f_0 = \SI{3.3}{\mega\hertz}
  \]
  \item Dobieramy kondensator o pojemności większej niż pojemność pasożytnicza tranzystora i mierzymy nową częstotliwość zakłóceń:
  \[
  C_1 = \SI{1}{\nano\farad}, \quad f_1 = \SI{2.7}{\mega\hertz}
  \]
  \item Liczymy stosunek częstotliwości:
  \[
  m = \frac{f_0}{f_1} = \frac{3.3}{2.7} = 1.22
  \]
  \item Obliczamy pojemność pasożytniczą tranzystora:
  \[
  C_0 = \frac{C_1}{m^2 - 1} = \frac{1\,\text{nF}}{1.22^2 - 1} = \SI{2.02}{\nano\farad}
  \]
  \item Obliczamy wartość indukcyjności pasożytniczej:
  \[
  L_0 = \frac{(m^2 - 1)}{(2\pi f_0)^2 C_1} = \frac{(1.22^2 - 1)}{(2\pi \cdot 3.3 \times 10^6)^2 \cdot 1 \times 10^{-9}} = \SI{1.15}{\micro\henry}
  \]
  \item Obliczamy minimalną wartość pojemności kondensatora snubbera:
  \[
  C_{\text{snubber}} = 3C_0 = 3 \cdot \SI{2.02}{\nano\farad} = \SI{6.06}{\nano\farad}
  \]
  \item Obliczamy wartość rezystancji w snubberze:
  \[
  R_{\text{snubber}} = \sqrt{\frac{L_0}{C_0}} = \sqrt{\frac{1.15 \times 10^{-6}}{2.02 \times 10^{-9}}} = \SI{24}{\ohm}
  \]
\end{enumerate}

\begin{figure}[H]
\centering
\includegraphics[width=0.8\textwidth]{aun1_rld_with_snubber_rgate100ohm.png}
\caption{Przebieg napiecia na wyjsciu tranzystora MOSFET sterujacego ukladem RLD z rezystancja bramki 100 [Ohm] i z tlumikiem oscylacji D-S}
\end{figure}

\begin{figure}[H]
\centering
\includegraphics[width=0.8\textwidth]{aun1_rld_with_snubber_rgate10ohm.png}
\caption{Przebieg napiecia na wyjsciu tranzystora MOSFET sterujacego ukladem RLD z rezystancja bramki 10 [Ohm] i z tlumikiem oscylacji D-S}
\end{figure}

\begin{table}[H]
\centering
\begin{tabular}{|l|c|c|}
\hline
\textbf{Tlumik} & \(\mathbf{T_{osc} \ [\mu s]}\) & \(\mathbf{A_{osc} \ [V]}\) \\
\hline
Bez tlumika, \(R_{gate} = 100\,\Omega\) & 1.0958 & 30.7943 \\
\hline
Bez tlumika, \(R_{gate} = 10\,\Omega\) & 1.8058 & 50.8134 \\
\hline
Tlumik, \(R_{gate} = 10\,\Omega\) & 1.0968 & 10.0267 \\
\hline
Tlumik, \(R_{gate} = 100\,\Omega\) & 1.1505 & 10.4941 \\
\hline
\end{tabular}
\caption{Porownanie napiecia na drenie tranzystora MOSFET sterujacym ukladem RLD z oraz bez uzycia tlumika}
\label{tab:oscillation_data}
\end{table}

\end{figure}

\section{Część II}

\subsection{TYPOWA REALIZACJA TORU STEROWANIA TRANZYSTORA MOSFET}

f) Badanie wplywu wartosci komponentow na strate mocy na tranzystorze. Porownanie dynamiki transoptora analogowego i cyfrowego z uwzglednieniem pradu diody. Zaleznosc wytracanej mocy na tranzystorze w zaleznosci od czestotliwosci PWM.

Zapoznano się ze strukturą i zamodelowano typowy tor sterowania tranzystorem MOSFET.
Układ składa się z bloku separacji galwanicznej – transoptora, układu wzmocnienia sygnału oraz gate drivera.
Transoptor zapewnia, że sygnał sterujący jest przekazywany bez elektrycznego połączenia, umożliwiając pracę w układzie o różnych punktach odniesienia masy i chroniąc elektronikę sterującą przed uszkodzeniem.
Gate driver odpowiada za szybkie i pewne przełączanie tranzystora poprzez odpowiednie ładowanie i rozładowywanie jego pojemności bramki, co ma istotny wpływ na straty przełączania oraz niezawodność pracy całego układu.
Do wejścia układu sterującego podawany jest sygnał PWM (ang. Pulse Width Modulation), który steruje pracą transoptora, a w konsekwencji – tranzystora MOSFET. Sygnał ten determinuje częstotliwość oraz czas włączenia tranzystora, wpływając bezpośrednio na charakterystykę napięcia i prądu w obwodzie wyjściowym.

Porównano dwa transoptory - analogowy (PC817) oraz cyfrowy (ISOM8710). 
Dodatkowo zbadano wpływ prądu diody na ich dynamikę i przedstawiono na wykresach 

\begin{figure}[H]
\centering
\includegraphics[width=0.8\textwidth]{aun1_gate_circuit_digital_vs_analog_rin100ohm.pdf}
\caption{Przebiegi napiecia na wyjsciu tranzystora MOSFET z transoptorem dla rezystancji wejsciowej 100 [Ohm]}
\end{figure}

\begin{figure}[H]
\centering
\includegraphics[width=0.8\textwidth]{aun1_gate_circuit_digital_vs_analog_rin230ohm.pdf}
\caption{Przebiegi napiecia na wyjsciu tranzystora MOSFET z transoptorem dla rezystancji wejsciowej 230 [Ohm]}
\end{figure}

\begin{figure}[H]
\centering
\includegraphics[width=0.8\textwidth]{aun1_gate_circuit_digital_vs_analog_rin500ohm.pdf}
\caption{Przebiegi napiecia na wyjsciu tranzystora MOSFET z transoptorem dla rezystancji wejsciowej 500 [Ohm]}
\end{figure}

\begin{figure}[H]
\centering
\includegraphics[width=0.8\textwidth]{aun1_gate_circuit_digital_vs_analog_rin1000ohm.pdf}
\caption{Przebiegi napiecia na wyjsciu tranzystora MOSFET z transoptorem dla rezystancji wejsciowej 1000 [Ohm]}
\end{figure}

\begin{table}[H]
\centering
\begin{tabular}{|l|c|c|}
\hline
\textbf{Rezystancja wejściowa} & \(\mathbf{I_{analog} \ [mA]}\) & \(\mathbf{I_{cyfrowy} \ [mA]}\) \\
\hline
100\,$\Omega$ & 18.0 & 14.0 \\
\hline
230\,$\Omega$ & 8.0 & 6.5 \\
\hline
500\,$\Omega$ & 3.8 & 3.2 \\
\hline
1000\,$\Omega$ & 1.9 & 1.6 \\
\hline
1560\,$\Omega$ & 1.3 & 1.0 \\
\hline
\end{tabular}
\caption{Porównanie prądu diody w funkcji rezystancji wejściowej dla sygnału analogowego i cyfrowego}
\label{tab:diode_current_comparison}
\end{table}

Z wykresów widać, że im mniejszy prąd na diodzie tym mniejsza minimalna wartość napięcia na transoptorze analogowym. Natomiast na transoptorze cyfrowym napięcie minimalne się nie zmienia, aż do momentu przekroczenia 1560 Ohm na rezystorze wejściowym - wtedy napięcie ma wartość niezmienną 5V.

Dynamika zmienia się także przy zmianie częstotliwości, co pokazano na wykresach:

\begin{figure}[H]
\centering
\includegraphics[width=0.8\textwidth]{aun1_gate_circuit_digital_vs_analog_5khz.pdf}
\caption{Przebiegi napiecia na wyjsciu tranzystora MOSFET z transoptorem dla sygnalu PWM o czestotliwosci 5kHz}
\end{figure}

\begin{figure}[H]
\centering
\includegraphics[width=0.8\textwidth]{aun1_gate_circuit_digital_vs_analog_20khz.pdf}
\caption{Przebiegi napiecia na wyjsciu tranzystora MOSFET z transoptorem dla sygnalu PWM o czestotliwosci 20kHz}
\end{figure}

Na wykresach widać, że transoptor analogowy nie radzi sobie tak dobrze jak transoptor cyfrowy dla wysokich częstotliwości.

\begin{table}[H]
\centering
\begin{tabular}{|l|c|c|}
\hline
\textbf{Częstotliwość} & \textbf{MOSFET IPW65R041CFD7 (W)} & \textbf{IGBT IKQB120N75CP2 (W)} \\
\hline
1\,kHz & -- & -- \\
\hline
10\,kHz & -- & -- \\
\hline
50\,kHz & -- & -- \\
\hline
100\,kHz & -- & -- \\
\hline
\end{tabular}
\caption{Średnia moc wytracana na tranzystorze w zależności od częstotliwości}
\end{table}

\subsection{MOSTKOWE UKŁADY WZMACZNIACZY TRANZYSTOROWYCH}

g) Problemy mostkow mocy

Mostek mocy to układ elektroniczny wykorzystywany do sterowania przepływem prądu przez obciążenie, najczęściej silnik. Najpopularniejszą konfiguracją jest tzw. mostek H, który składa się z czterech tranzystorów mocy. Dzięki odpowiedniemu przełączaniu tych tranzystorów możliwe jest sterowanie kierunkiem oraz wartością prądu płynącego przez obciążenie. Mostki mocy znajdują zastosowanie w układach napędowych i przekształtnikach, a ich działanie opiera się na szybkim przełączaniu elementów półprzewodnikowych w celu efektywnego zarządzania energią.

W mostku H kluczowe są 3 rodzaje sterowania:
-- unipolarny
-- quasi-bipolarny
-- bipolarny

h) Analiza dzialania ukladu sterowania bipolarnego, unipolarnego, quasi-bipolarnego

W układach z obciążeniem RLE (rezystancja, indukcyjność i źródło siły elektromotorycznej), istotnym zjawiskiem jest konieczność zapewnienia ścieżki demagnetyzacji dla prądu cewki. Podczas wyłączania tranzystorów sterujących, energia zgromadzona w indukcyjności nie może zostać natychmiast rozproszona. W takim przypadku pojawiają się przepięcia i zakłócenia, o ile nie zostanie zapewniona odpowiednia ścieżka dla tego prądu.
Sterowanie unipolarne:
 W tym trybie tranzystory włączane są na przemian tylko w jednej gałęzi mostka (np. górnej). Prąd powrotny z cewki płynie przez diody swobodne w dolnej gałęzi. Ścieżka demagnetyzacji realizowana jest wyłącznie przez te diody, co powoduje wolniejszą dynamikę wyłączania i dłuższy czas ustalania się prądu.

Sterowanie quasi-bipolarne:
Jest to tryb pośredni, gdzie w każdej połowie okresu włączany jest jeden tranzystor, natomiast przeciwny tranzystor w drugiej gałęzi jest wyłączony. Demagnetyzacja cewki odbywa się przez diody swobodne przeciwległej gałęzi, które stanowią naturalną ścieżkę dla prądu indukowanego. Dzięki temu dynamika pracy jest lepsza niż w trybie unipolarnym, a układ zachowuje prostotę sterowania i mniejsze straty energetyczne.

Sterowanie bipolarne:
Tranzystory włączane są naprzemiennie w obu gałęziach mostka, co pozwala na aktywne sterowanie przepływem prądu w obydwu kierunkach. W momencie przełączania prąd z cewki jest wymuszany przez aktywnie sterowany tranzystor w drugiej gałęzi, co zapewnia najszybszą i najbardziej precyzyjną ścieżkę demagnetyzacji. Jednak kosztem wyższych strat mocy i bardziej skomplikowanego układu sterowania.

i) Model bipolarnego i unipolarnego sterowania

\begin{figure}[H]
\centering
\includegraphics[width=0.8\textwidth]{aun1_bipolar_bridge.png}
\caption{Przebieg napiecia na silniku przy mostku sterowanym bipolarnie}
\end{figure}

\begin{figure}[H]
\centering
\includegraphics[width=0.8\textwidth]{aun1_unipolar_bridge.png}
\caption{Przebieg napiecia na silniku przy mostku sterowanym unipolarnie}
\end{figure}

\begin{figure}[H]
\centering
\includegraphics[width=0.8\textwidth]{aun1_bipolar_bridge2.png}
\caption{Przebieg predkosci na silniku przy mostku sterowanym bipolarnie}
\end{figure}

\begin{figure}[H]
\centering
\includegraphics[width=0.8\textwidth]{aun1_unipolar_bridge2.png}
\caption{Przebieg predkosci na silniku przy mostku sterowanym unipolarnie}
\end{figure}

Z wykresów wynika, że przy sterowaniu unipolarnym dynamika układu jest gorsza, a odpowiedź ustala się wolniej. Sterowanie to jest jednak bardziej efektywne energetycznie.

\section{Część III}

\subsection{BADANIA EKSPERYMENTALNE}

j) Przebiegi prądów oraz napięć na stanowiskach eksperymentalnych

Uwaga: brak jednostek na osi OY spowodowany jest faktem, ze dane
zapisane byly w formacie raw, niezeskalowane

Przebiegi na stanowisku Alspa- amplituda napiecia 250 [V], amplituda pradu 1.5 [A], regulacja predkosci odbywala sie przez zmiane czestotliwosci napiecia i pradu

\begin{figure}[H]
\centering
\includegraphics[width=0.8\textwidth]{aun1_alspa_rpm300.pdf}
\caption{Przebieg napiecia oraz pradu na stanowisku Alspa przy predkosci katowej 300 [RPM]}
\end{figure}

\begin{figure}[H]
\centering
\includegraphics[width=0.8\textwidth]{aun1_alspa_rpm600.pdf}
\caption{Przebieg napiecia oraz pradu na stanowisku Alspa przy predkosci katowej 600 [RPM]}
\end{figure}

\begin{figure}[H]
\centering
\includegraphics[width=0.8\textwidth]{aun1_alspa_rpm900.pdf}
\caption{Przebieg napiecia oraz pradu na stanowisku Alspa przy predkosci katowej 900 [RPM]}
\end{figure}

Przebiegi na stanowisku Microverter

\begin{figure}[H]
\centering
\includegraphics[width=0.8\textwidth]{aun1_microverter_rpm300.pdf}
\caption{Przebieg napiecia oraz pradu na stanowisku Microverter przy predkosci katowej 300 [RPM]}
\end{figure}

\begin{figure}[H]
\centering
\includegraphics[width=0.8\textwidth]{aun1_microverter_rpm600.pdf}
\caption{Przebieg napiecia oraz pradu na stanowisku Microverter przy predkosci katowej 600 [RPM]}
\end{figure}

\begin{figure}[H]
\centering
\includegraphics[width=0.8\textwidth]{aun1_microverter_rpm900.pdf}
\caption{Przebieg napiecia oraz pradu na stanowisku Microverter przy predkosci katowej 900 [RPM]}
\end{figure}

Przebiegi napiecia i pradu na stanowisku Unidrive

\begin{figure}[H]
\centering
\includegraphics[width=0.8\textwidth]{aun1_unidrive_rpm300.pdf}
\caption{Przebieg napiecia oraz pradu na stanowisku Unidrive przy predkosci katowej 300 [RPM]}
\end{figure}

\begin{figure}[H]
\centering
\includegraphics[width=0.8\textwidth]{aun1_unidrive_rpm600.pdf}
\caption{Przebieg napiecia oraz pradu na stanowisku Unidrive przy predkosci katowej 600 [RPM]}
\end{figure}

\begin{figure}[H]
\centering
\includegraphics[width=0.8\textwidth]{aun1_unidrive_rpm900.pdf}
\caption{Przebieg napiecia oraz pradu na stanowisku Unidrive przy predkosci katowej 900 [RPM]}
\end{figure}

Przebiegi na stanowisku DML

Uwaga: Niestety, ale grupa od ktorej otrzymalismy te pomiary (prowadzacy zgodzil sie na taki proceder, poniewaz zadna z grup nie zdazyla wykonac eksperymentu na kazdym stanowisku) nie zapisala dla jakich predkosci katowych
byly robione pomiary

\begin{figure}[H]
\centering
\includegraphics[width=0.8\textwidth]{aun1_dml_obciazenie_weak.pdf}
\caption{Przebieg napiecia oraz pradu na stanowisku Unidrive przy malym obciazeniu}
\end{figure}

\begin{figure}[H]
\centering
\includegraphics[width=0.8\textwidth]{aun1_dml_obciazenie_medium.pdf}
\caption{Przebieg napiecia oraz pradu na stanowisku Unidrive przy srednim obciazeniu}
\end{figure}

\begin{figure}[H]
\centering
\includegraphics[width=0.8\textwidth]{aun1_dml_obciazenie_hard.pdf}
\caption{Przebieg napiecia oraz pradu na stanowisku Unidrive przy duzym obciazeniu}
\end{figure}


\section{Część IV}

\subsection{MOSTEK TYRYSTOROWY}

\end{document}
